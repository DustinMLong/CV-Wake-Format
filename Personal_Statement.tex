\hskip 1cm I am an Assistant Professor in the Department of Biostatistics and Data Science in the Division of Public Health Sciences at Wake Forest University School of Medicine (WFUSM). My research interests include machine learning, risk prediction, cardiovascular disease, blood pressure, and hypertension. I have taught introductory R programming classes for several years, and have developed curriculum for remote learning that allows students to engage in team science and creative problem instead of long lectures about code syntax. I have also mentored multiple doctoral students in both Biostatistics and Epidemiology and served as a member on their dissertation committee. 


I received my PhD in Statistics at the University of Chapel Hill in North Carolina. My dissertation developed an $R^2$ statistics for the generalized linear mixed model. Specifically, I developed $R^2$ statistics for both the fixed effects in the mixed model and random effects in the mixed model. The papers I published from this work have been collectively cited over 300 times since 2017, and the software I developed and maintain (\texttt{r2glmm}) has been downloaded over 40,000 times. This research required me to learn a great deal about the R programming language, R package development, and statistical computing, which have become the foundation of the work that I do today.

% I spent four years in the Department of Biostatistics at the University of Alabama at Birmingham (UAB) School of Public Health and joined the WFUSM faculty in 2021. At Birmingham, I was an early stage investigator in the Jackson Heart Study Hypertension Working Group and a Biostatistician in the American Heart Association Strategically Focused Research Network and the Kirklin Institute for Research in Surgical Outcomes. In these roles, I became an expert in blood pressure, hypertension, cardiovascular disease, and machine learning. 

I taught four introductory R programming classes while I was at UAB (one per year). The size of my class was consistently 10 or more students, and my teaching evaluations improved each year. During my summer 2020 course (i.e., the first semester of fully remote learning due to the pandemic), I completely redesigned my curriculum to make it compatible with remote learning. In addition, I built a complete website (\url{http://bst680.rbind.io/syllabus/}) that students could use to find resources, lecture notes, homework exercises, and interactive tutorials for course content. Most importantly, I put students into breakout rooms during class and gave them challenging programming tasks to work through instead of giving long lectures. The comments I received from students about this design choice were strongly appreciative, and many of them found they learned more about R by engaging with these problems. Overall, students have rated my courses very well, with my average teaching evaluation scores consistently placing me between the 85th and 95th percentile of all courses in the US that were rated using the Individual Development and Educational Assessment survey.

While at UAB, I researched the random forest and engaged in collaborative research investigating blood pressure and cardiovascular disease. I published an article developing the oblique random survival forest, an extension of the random forest that demonstrated excellent prediction accuracy compared to standard random forests and other machine learning models for right-censored outcomes. The oblique random survival forest has been recognized for its high prediction accuracy, with a recent paper in Circulation finding that the oblique random survival forest provided more accurate predictions for incident heart failure compared to published risk prediction equations and state-of-the-art machine learning algorithms. In my collaborative research, my colleagues and I found blood pressure control had declined among adults with hypertension from 2013 to 2018. The article we wrote summarizing these findings was published in the Journal of the American Medical (JAMA) Association, where it has been cited over 325 times since 2020. 

\AR{At the Wake Forest University School of Medicine (WFUSM), I have become the principal investigator (PI) of two pilot awards focused on improving the computational efficiency and interpretability of oblique random survival forests. To accomplish these aims, I have developed an R package, \texttt{aorsf}, that matches the computational efficiency of professional-grade R packages such as \texttt{ranger}, \texttt{party}, and \texttt{randomForestSRC}, with novel methods to fit and interpret oblique random survival forests. I have presented this work to the Public Health Science division and the Bioinformatics Institute at Wake Forest, where I was awarded with an early stage investigator award. In addition to this award, \texttt{aorsf} received the first gold badge awarded to a statistical software package from \texttt{rOpenSci}, an online platform that performs software review aimed at quality improvement. I recently published the award winning software design of \texttt{aorsf} in the Journal of Open Source Software and have submitted a separate paper detailing its methodology to the Journal of Machine Learning.}

\AR{In addition to my statistical research, I am the PI of two R01 sub-contracts awarded to WFUSM entitled ``Kidney Tubule Dysfunction and Future Risk of Acute Kidney Injury" and ``Incorporation of a Hypertension Working Group into the Jackson Heart Study". In the former, I have contributed to data collection and quality control during the first year of a four year study. In the latter, I have helped develop and publish high impact papers on the etiology of hypertension and cardiovascular disease, including two recent papers entitled ``Inflammation biomarkers and incident coronary heart disease: the Reasons for Geographic And Racial Differences in Stroke Study" and ``Prevalence, risk factors, and cardiovascular disease outcomes associated with persistent blood pressure control: The Jackson Heart Study".}

\AR{My collaborative research network has grown since I joined WFUSM. As a co-I in the Molecular Transducers of Physical Activity Consortium (MoTrPAC), I am developing an R package to help investigators apply standardized data processing pipelines to our accelerometry data in a secure and reproducible manner. As a co-I in the SPRINT MIND study, I am the lead author for an investigation of long-term mortality for participants in the Systolic blood PRessure INtervention Trial (SPRINT), which has been accepted for publication in a high impact journal (I am not allowed to disclose the journal name yet).}
